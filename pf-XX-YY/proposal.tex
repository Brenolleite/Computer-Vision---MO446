\documentclass[12pt,a4paper]{article}
\usepackage[utf8]{inputenc} % sempre salve seus arquivos como UTF8
\usepackage[T1]{fontenc}
\usepackage[english]{babel}

\usepackage[left=2.5cm,right=2cm,top=2cm,bottom=2.5cm]{geometry}
\usepackage{amsmath}
\usepackage{amsthm}
\usepackage{amsfonts}

\usepackage{graphicx}
\usepackage{algorithm}
\usepackage{color}
\usepackage[noend]{algpseudocode}
\usepackage{mathtools}
\usepackage{subfig}
\usepackage{diagbox}

% load times font
\usepackage{mathptmx}
\usepackage[scaled=.90]{helvet}
\usepackage{courier}

% comandos
\newcommand{\mdc}[1]{\mathrm{mdc}(#1)}

\DeclarePairedDelimiter\ceil{\lceil}{\rceil}
\DeclarePairedDelimiter\floor{\lfloor}{\rfloor}

% Foot without marker
\newcommand\blfootnote[1]{%
	\begingroup
	\renewcommand\thefootnote{}\footnote{#1}%
	\addtocounter{footnote}{-1}%
	\endgroup
}

\title{MO446 -- Introduction to Computer Vision  \\ Object Tracking Proposal}
\author{Breno Leite  \\ Guilherme Leite}
\date{05/10/2017}

\begin{document}

\maketitle

%% ---------------- Starts here --------------------------------

\textbf{\LARGE 1 - Proposal} \\

	In the final project we propose to solve a ball tracking problem. Our baseline is to track a single coloured ball. And for more experiments we will extend the problem to several balls, which contains different colors. In this proposal we want to deal with tracking different balls, even when occlusions and/or swaps between positions happen. We propose to use classical algorithms to solve the problem, like KLT, Kalman filter, color matching, and circle Hough transform. \\
	
\textbf{\LARGE 2 - Dataset} \\

	We will create a dataset for the problem, which will be videos taken on a smooth light coloured plane surface and controlled illumination with a single still cellphone camera. The setup conditions will be changed in order to test our solution's resilience.\\

\textbf{\LARGE 3 - Related Work} \\

	The following papers might be used as basis of this project. Even though they are not directly related to our proposal, we believe they will provide insights and tools to implement our solution.
	A method to track an object using its dominant colour and Kalman filter is shown on [1]. Moreover, in [2] a up to date survey on object detection and tracking can be found. In another paper [3], a single low resolution camera and particle filters are used to estimate a tennis ball position in the field.\\
	
\newpage
	
\centerline{\textbf{\LARGE References}}

[1] - Shiuh-Ku Weng, Chung-Ming Kuo, Shu-Kang Tu, "Video object tracking using adaptive Kalman filter", Journal of Visual Communication and Image Representation 2006, no. 17, pp. 1190-1208. \\

[2] - Dixit, Astha, Manoj Verma, and Kailash Patidar. "Survey on Video Object Detection and Tracking." International Journal of Current Trends in Engineering and Technology 2.02 (2016). \\ 

[3] - Yan, F, Christmas, W and Kittler, J (2005) A Tennis Ball Tracking Algorithm for Automatic Annotation of Tennis Match In: British Machine Vision Conference, 2005-09-05 - 2005-09-08, Oxford, UK. \\

\end{document}
