\documentclass[12pt,a4paper]{article}
\usepackage[utf8]{inputenc} % sempre salve seus arquivos como UTF8
\usepackage[T1]{fontenc}
\usepackage[english]{babel}

\usepackage[left=2.5cm,right=2cm,top=2cm,bottom=2.5cm]{geometry}
\usepackage{amsmath}
\usepackage{amsthm}
\usepackage{amsfonts}

\usepackage{graphicx}
\usepackage{algorithm}
\usepackage{color}
\usepackage[noend]{algpseudocode}
\usepackage{mathtools}
\usepackage{subfig}
\usepackage{diagbox}

% load times font
\usepackage{mathptmx}
\usepackage[scaled=.90]{helvet}
\usepackage{courier}

% comandos
\newcommand{\mdc}[1]{\mathrm{mdc}(#1)}

\DeclarePairedDelimiter\ceil{\lceil}{\rceil}
\DeclarePairedDelimiter\floor{\lfloor}{\rfloor}

% Foot without marker
\newcommand\blfootnote[1]{%
	\begingroup
	\renewcommand\thefootnote{}\footnote{#1}%
	\addtocounter{footnote}{-1}%
	\endgroup
}

\title{MO446 -- Introduction to Computer Vision  \\ Proposal}
\author{Breno Leite  \\ Guilherme Leite}
\date{05/10/2017}

\begin{document}

\maketitle
\blfootnote{\textit{\textbf{Important note:} The borders seen in the figures are not part of the image, they are figurative information about the starting and ending points of the image. Moreover, all the image scales in this report were changed in order to make the text more readable.}} \\

%% ---------------- Starts here --------------------------------

\textbf{\LARGE 1 - Proposal} \\

	To the final project we propose to solve a ball tracking problem. Our baseline is to track a single coloured ball. Upon completion, the problem will be extended to a couple balls, several balls, different colours balls, same colour balls, slow moving ball and fast moving ball.\\
	
\textbf{\LARGE 2 - Dataset} \\

	The dataset to be used in the experimentation will be of our own making. A smooth light coloured plane surface as background with controlled illumination, the variables of this setup will be changed to test the solution's resilience.\\

\textbf{\LARGE 3 - Related Work} \\

	The following papers were reviewed to be used as basis of this project. Even though they are not directly related to our proposal, we believe they will provide insights and tools to implement our solution.
	\par[1] has a method to follow a object using its dominant colour and uses Kalman filter to track it. [2] performs a up to date survey around object detection and tracking papers. [3] uses a single low resolution camera and particle filters to estimate the tennis ball position in the field.\\
	
\textbf{\LARGE 4 - References} \\

	\par [1] - Shiuh-Ku Weng, Chung-Ming Kuo, Shu-Kang Tu, "Video object tracking using adaptive Kalman filter", Journal of Visual Communication and Image Representation 2006, no. 17, pp. 1190-1208.
	\par [2] - Dixit, Astha, Manoj Verma, and Kailash Patidar. "Survey on Video Object Detection and Tracking." International Journal of Current Trends in Engineering and Technology 2.02 (2016).
	\par [3] - Yan, F, Christmas, W and Kittler, J (2005) A Tennis Ball Tracking Algorithm for Automatic Annotation of Tennis Match In: British Machine Vision Conference, 2005-09-05 - 2005-09-08, Oxford, UK.

\end{document}
