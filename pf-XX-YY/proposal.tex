\documentclass[12pt,a4paper]{article}
\usepackage[utf8]{inputenc} % sempre salve seus arquivos como UTF8
\usepackage[T1]{fontenc}
\usepackage[english]{babel}

\usepackage[left=2.5cm,right=2cm,top=2cm,bottom=2.5cm]{geometry}
\usepackage{amsmath}
\usepackage{amsthm}
\usepackage{amsfonts}

\usepackage{graphicx}
\usepackage{algorithm}
\usepackage{color}
\usepackage[noend]{algpseudocode}
\usepackage{mathtools}
\usepackage{subfig}
\usepackage{diagbox}

% load times font
\usepackage{mathptmx}
\usepackage[scaled=.90]{helvet}
\usepackage{courier}

% comandos
\newcommand{\mdc}[1]{\mathrm{mdc}(#1)}

\DeclarePairedDelimiter\ceil{\lceil}{\rceil}
\DeclarePairedDelimiter\floor{\lfloor}{\rfloor}

% Foot without marker
\newcommand\blfootnote[1]{%
	\begingroup
	\renewcommand\thefootnote{}\footnote{#1}%
	\addtocounter{footnote}{-1}%
	\endgroup
}

\title{MO446 -- Introduction to Computer Vision  \\ Proposal}
\author{Breno Leite  \\ Guilherme Leite}
\date{05/10/2017}

\begin{document}

\maketitle
\blfootnote{\textit{\textbf{Important note:} The borders seen in the figures are not part of the image, they are figurative information about the starting and ending points of the image. Moreover, all the image scales in this report were changed in order to make the text more readable.}} \\

%% ---------------- Starts here --------------------------------

\textbf{\LARGE 1 - Proposal} \\

1 bolinha tracking - 2 bolinhas tracking - varias bolinhas tracking - 2 bolinhas mesma cor sem confundir\\
	To the final project we propose a escalation of a ball tracking implementation, at first a single coloured ball tracking, once completed it will be escalated to a couple different coloured balls tracking that finally will be escalated to multiple different coloured fast ball tracking \\
	
\textbf{\LARGE 2 - Dataset} \\

propria dataset numa superficie clara, sempre o mesmo background, iluminacao controlada como baseline - mais pra frente mudar essas variaveis\\

\textbf{\LARGE 3 - Related Work} \\

\end{document}
