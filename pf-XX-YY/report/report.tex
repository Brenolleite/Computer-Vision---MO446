\documentclass[10pt,twocolumn,letterpaper]{article}

\usepackage{cvpr}
\usepackage{times}
\usepackage{epsfig}
\usepackage{graphicx}
\usepackage{amsmath}
\usepackage{amssymb}

% Include other packages here, before hyperref.

% If you comment hyperref and then uncomment it, you should delete
% egpaper.aux before re-running latex.  (Or just hit 'q' on the first latex
% run, let it finish, and you should be clear).
\usepackage[breaklinks=true,bookmarks=false]{hyperref}

\cvprfinalcopy{} % *** Uncomment this line for the final submission

\def\cvprPaperID{****} % *** Enter the CVPR Paper ID here
\def\httilde{\mbox{\tt\raisebox{-.5ex}{\symbol{126}}}}

% Pages are numbered in submission mode, and unnumbered in camera-ready
%\ifcvprfinal\pagestyle{empty}\fi
\setcounter{page}{1}
\begin{document}

%%%%%%%%% TITLE
\title{Classic Methods on Color Based Ball Tracking}

\author{Breno Leite\\ Guilherme Leite\\
Instituto de Computa\cc\~ao --- UNICAMP\\
Cidade Universit\'aria, Campinas/SP\\
{\tt\small brenolleite@gmail.com\\ \tt\small guilherme.vieira.leite@gmail.com}
% For a paper whose authors are all at the same institution,
% omit the following lines up until the closing ``}''.
% Additional authors and addresses can be added with ``\and'',
% just like the second author.
% To save space, use either the email address or home page, not both
% \and
% Guilherme Leite\\
% Instituto de Computa\cc\~ao --- UNICAMP\\
% Cidade Universit\'aria, Campinas/SP\\
% {\tt\small guilherme.vieira.leite@gmail.com}
}

\maketitle
%\thispagestyle{empty}

%%%%%%%%% ABSTRACT
\begin{abstract}
  Abstract
\end{abstract}

%%%%%%%%% BODY TEXT
\section{Introduction}

  Ball tracking is a classical problem present in a diverse range of
  applications. Nowadays it is used in sports events to automatically track the
  focus of the action and game score, it is also widely used in robotics,
  specially in robocup’s soccer competitions.

%-------------------------------------------------------------------------

\section{Related Work}

  Related work

%-------------------------------------------------------------------------
\section{Methodology}

  Initially the tracking was done alone by a color detector. Using the color
  space HSV to select a color range that would create a mask, in this mask every
  pixel in the range is set to white and everything else black, afterwards two
  erosions and dilations are applied to reduce noise. The connected components
  are extracted from the resulting mask, and the largest connected component is
  selected as a region that contains a colored ball of such range. Every other
  connected component region with size of at least 30\% of the largest one is
  also selected as another object. Each of these selected regions has a center
  of mass, the coordinate of this point is used as a rough estimate of
  the ball's position and is taken as the tracking parameter, later on this
  information will be called `tracking position'.

  The previous steps should insure the detection of a colored blue, but it can't
  distinguish between a ball and a cube. The Hough Circle Transform is used in
  the mask obtained previously, and it tries to fit circles in the mask, using
  the edges detected by its internal canny filter. To ensure the best fit the
  parameter SUCH, WHAT SUCH DOES, is started with a value high enough to not
  find a single circle in the mask, from this high value it is decreased by one
  until it finds its first circle, the region below this circle is selected as a
  ball region. This process is repeated until up to $K$ balls are fit. Using
  Hough Circle Transform the tracking narrows down to track only the roughly
  rounded shapes in the mask.

  Applying the color detection every frame is redundant, add hough transform to
  that and it becomes expensive and redundant. Hough is applyed every $N$ frame,
  in each of these frames the position of the ball is corrected and from then it
  is estimated by the difference of its velocity and direction between two
  frames. These steps are faster, enabling the overall solution to perform in
  real time.

  LK motion flow is enough to estimate the ball position, but it always depends
  on the last frame. Add a scenario in which the ball is occluded for more than
  a couple frame and LK motion will completely lose its estimative. Kalman
  filter is used to keep up with those scenarios.

%-------------------------------------------------------------------------
\section{Experiments}

  The experiments were designed to highlight the effect of each new feature that
  was incrementally added.

  Start
  We approached the scenarios listed above in a incremental
complexity manner.
Single colored ball detection:
Different colored balls detection:
Same colored balls detection:

Object detection without Hough Transform:
Object detection with Hough Transform:
Lucas Kanade Motion Flow:
Limitations of the Motion Flow and Hough Transform:
Kalman Filter Occlusion:
Real world application:

%-------------------------------------------------------------------------
\section{Results}

  Results

  HOUGH too slow, too much per frame
  Kalman filter can't keep up if the trajectory changes mid occlusion

%-------------------------------------------------------------------------
\section{Conclusion}

  Even though the classic methods were able to track and overcome some of the
  challenges presented in this report they could't keep their precision in a real world
  scenario. Some other approaches to detect and track are needed for scenarios
  like a volleyball match, where the ball is white and the people watching the
  game are often caught in the camera along side with the game. Other works used deep
  learning to train a network into detecting the ball, some used a physics model
  to calculate the parabolic trajectory of the ball and some used completely
  different clues, like the fact that a baskball player moves differently from
  the others when he/she pocess the ball.

  To go through the classic methods many of the course's topics were covered and
  applyed, clearing our sights towards their weakness, strenght and how they are
  applied. It is instigating to think the power that a neural network would have
  in such context, and realize from other works the vast number of ways to
  tackle a single problem.

{\small
\bibliographystyle{ieee}
\bibliography{egbib}
}

\end{document}
